\documentclass{article}

\title{Evergreen Entrance Essay}
\author{Paul Pham}

\begin{document}

\maketitle

In my first year of teaching at Evergreen, I am beginning many new projects. One project is to begin 
a new chapter of my life, and it has a very
personal goal: to help me adjust to a new role, a
new city, and a new way of thinking about myself.
This project is the transition to being a teacher,
or presenting myself as a teacher to the outside world.
Another project is to explore a new way of teaching,
a hybrid model that is halfway between a school
and a startup. Like many new ideas, it is not really
new, and it is something that I can only scratch
the surface of in my first year. However, it
feels good to have a goal, even if you change your mind in mid-stream. These two projects will feed
one another as I continue to grow as a human being
and as I come to realize my place as a world citizen.

Project Number 1: So you want to be a teacher.
I have been a student for
a long time, and I hope I will be a student and
a learner for the rest of my life. I became a
graduate student, or as my friend Ken Brown calls it,
the 17th grade, without a lot of thought, without
realizing that this is as close to being a
professional student as you can get. This lasted up
until the 21st grade, which is
how long it took me to get through grad school, with
a long three year break in the middle and lots of
wandering sidequests to Europe, working at Amazon,
being unemployed, employing myself off Craigslist,
and trying to find myself at Burning Man.
My interests changed a lot, and I realized I didn't
really like research, or research culture, which 
ironically is the main point of grad school. So how 
to get out, as quickly
and expeditiously as possible? How to go back to
being an amateur student and a lover of knowledge?
I came to being a teacher accidentally, in my
last year of grad school, when I had a chance to
design my own class from scratch. It was a mix of
graphic novel, quantum physics, computer science,
and historical fan fiction. This happened
at the same time that I was applying for jobs, and
I remember that Evergreen was the only school where
I was not afraid to reveal all of my weird,
interdisciplinary ideas. By that time, I was really
tired of pretending to be someone that I was not.
But what was I going to do with my newfound freedom?
What could I do as a professional teacher that I
couldn't do as a professional student?

Project Number 2: One thing that frustrated me about
grad school was the lack of feedback. At first I
thought it was a character flaw, that I needed
positive reinforcement and a connection between
effort and reward. Research was very uncertain
and self-guided, and the reward of getting a paper
published or contributing something new to
human knowledge was too rare and fleeting to feel
compelling to me. When I was unemployed, I made
websites for people on Craigslist, which is the
default programmer job. It is the fast food of
software, and it is as unprestigious as you can get
among programmers.
But I interacted with people on a face-to-face basis,
and every day was like a new adventure: what job was
I going to accept, where would it take me, who
would I meet?
What would I get to learn, could I even do it?
It wasn't a sustainable option, but it made money,
which I felt very physically as a learning experience. On the other hand, one of the things
that frustrated me the most about academia was
constantly applying for grants, which felt like
begging.
Was there a way to capture this empowering feeling
of being financially sustainable
and to share it with students? I envisioned a
class where I taught students to develop software
for paying clients, where they would experience
both the theoretical pleasure of thinking great
thoughts but also get a concrete thrill from
applying their skills to a real-world problem.

Both of these projects are very long-term visions.
It's not clear how much progress I will make during
the busy first year of my new job. However, in this
new chapter of my life, I want to choose my actions
deliberately to express my values, and not blindly
work out of inertia or to meet societal expectations.
I think an experiential, education-based startup
captures my values right now, and I think it will
fit in well with the values and interests of
my colleagues and students at Evergreen.
\end{document}