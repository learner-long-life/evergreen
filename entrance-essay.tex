\documentclass{article}

\usepackage{setspace}
\doublespacing

\title{Evergreen Entrance Essay}
\author{Paul Pham}

\begin{document}

\maketitle

In my first year at Evergreen, I am beginning many new projects. One project is
personal: to help me adjust to a new role, a
new city, and a new way of thinking about myself.
This project is the transition to being a teacher,
and presenting myself as a teacher to the outside world.
Another project is to explore a new way of teaching,
a hybrid model that is halfway between a school
and a startup. It
feels good to have a goal, even if you change your mind in mid-stream.
%or can't make much progress on it
%right away.
These two projects will feed
one another as I continue to realize my place as a
citizen in a larger community.
%, and eventually
%include the whole world.

\textbf{Project Number One:
So you want to be a teacher.}
%I have been a student for
%a long time, and I hope I will be a student and
%a learner for the rest of my life.
I became a
graduate student, or as my friend Ken Brown calls it,
the 17th grade, without a lot of thought. This 
lasted for a
a long three year break in the middle and lots of
wandering sidequests to Europe,
being unemployed, employing myself off Craigslist,
and trying to find myself at Burning Man.
I realized I didn't
really like research culture, which 
ironically is the main point of grad school, to
be a \emph{professional student}.
In my
last year of grad school, I had a chance to
design my own class from scratch. It was a mix of
graphic novel, quantum physics, computer science,
and historical fan fiction. This happened
at the same time that I was applying for
academic jobs, and
I remember that Evergreen was the only school where
I was not afraid to reveal all of my
interdisciplinary ideas. By that time, I was really
tired of pretending to be someone that I was not.
I want to be an \emph{amateur student}, one who
loves learning, and I thought that being a
\emph{professional teacher} was the way to do it.
But what was I going to do with my newfound freedom
that I couldn't do as a professional student?

\textbf{Project Number Two: Only personal experience teaches.}
One thing that frustrated me about
grad school was the lack of feedback. At first I
thought it was a character flaw, that I needed
positive reinforcement and a connection between
effort and reward. Research was very uncertain
and self-guided, and the reward of getting a paper
published
% or contributing something new to
%human knowledge
was too rare and fleeting to feel
compelling to me. When I was unemployed, I made
websites for people on Craigslist, which is the
default programmer job. It is the fast food of
software, and it is as unprestigious as you can get
among programmers.
But I interacted with clients on a face-to-face basis,
and every day was like a new adventure: what job was
I going to accept, where would it take me, who
would I meet?
What would I get to learn, could I even do it?
%It wasn't a sustainable long-term option, but it made money,
%which I felt very physically as a learning experience.
%On the other hand, one of the things
%that frustrated me most about academia was
%constantly applying for grants and fearing for
%survival.
Was there a way to capture this empowering feeling
of being financially sustainable without the
emotional downside,
and to share it with students? I envisioned a
class where I taught students to develop software
for paying clients, where they would experience
both the theoretical pleasure of thinking great
thoughts but also get a concrete thrill from
applying their skills to a real-world problem.

Both of these projects are long-term visions.
It's not clear how much progress I will make during
the busy first year of my new job. However, in this
new chapter of my life, I want to choose my actions
deliberately to express my values, and not blindly
work out of inertia or to meet others' expectations.
I think an experiential, education-based startup
captures my values right now, and I think it will
fit in well with the values and interests of
my colleagues and students at Evergreen.

\end{document}